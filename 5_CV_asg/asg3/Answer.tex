\documentclass[12pt, a4paper]{exam}
\usepackage{graphicx}
\usepackage[left=0.8in, top=0.7in, total={6.2in,10in}]{geometry}
\usepackage[normalem]{ulem}
\usepackage{comment}
\usepackage{hyperref}
\usepackage{float}

\renewcommand\ULthickness{1.0pt}   %%---> For changing thickness of underline
\setlength\ULdepth{1.3ex}%\maxdimen ---> For changing depth of underline

\begin{document}
	%\thispagestyle{empty}
	\noindent
	\begin{minipage}[l]{0.1\textwidth}
		\noindent
		\includegraphics[width=1.8\textwidth]{res/iiserb_logo.png}
	\end{minipage}
\hfill
\begin{minipage}[c]{0.8\textwidth}
	\begin{center}
		{\large	Indian Institute of Science Education and Research Bhopal \par
		\large	\par
	\large \textbf{	Computer Vision(DSE-312/EECS-320)}	\par
\small	Assignment-2}
	\end{center}
\end{minipage}

\par
\vspace{0.2in}
\noindent
\textbf{Name: }Adheesh Trivedi\\
\noindent
\textbf{Roll No.:  }22016\\
\noindent
\uline{\textbf{Time of submission: } \hfill 		\hfill Marks Obtained: } \\
\uline{Please follow the instructions given in the assignment carefully.}
\par 
\vspace{0.15in}
\noindent
\centering
{\small \bfseries
\begin{enumerate}
    \item All questions are mandatory. Plagiarism and copying from anywhere (similar
    submission) can debar you from this course and invite the academic dishon-
    esty policy.
    \item Implement all algorithms purely in Python without using specialized libraries
    like OpenCV or PIL for the processing. You may use libraries for basic
    operations (like loading an image), but the algorithms should be coded from
    scratch.
    \item Comment on your code extensively to explain your logic and the steps you
    are implementing.
    \item Display both the original and processed images to compare results.
    \item Make a short 7-minute video and explain your code.
    \item A report reflecting on what you have learned. Visualization of the output
    must be there along with other necessary details.
\end{enumerate}
}

\vspace{0.2in}
\begin{questions}
	\pointsdroppedatright
	\question

Apply the filters mentioned below on the image attached and analyze their impact.
Describe what you found after applying each filter and why certain phenomena
occur. \textbf{\textit{(Marks:6)}}
\begin{itemize}
    \item SIFT
    \item Bag of Words
    \item HOG
\end{itemize}

\begin{figure}[H]
    \centering
    \includegraphics[width=.3\textwidth]{res/Albert_Einstein.jpg}
    \caption{Image of Albert Einstein}
    \label{fig:1q}
\end{figure}

\vspace{0.2in}
	\pointsdroppedatright 
\textbf{Answer:}

\textbf{Computing Integral Image}

I have computed integral image using the following algorithm

\[
II[i, j] = I[i, j] + II[i-1, j] + II[i, j-1] - II[i-1, j-1]
\]

where I is the original image and II is the integral image.

\textbf{Haar-like features}

I have used the four type of Haar-like features as shown in the figure below.

\begin{figure}[H]
    \centering
    \includegraphics[width=.7\textwidth]{imgs/haar_like_features.png}
    \caption{Haar-like features}
    \label{fig:2_mediapipe}
\end{figure}

\textbf{Face Detection}

I have seletected first, second and fourth Haar-like features to detect face. I have used the following algorithm to detect face.

\begin{itemize}
    \item Check if the first haar-like feature evaluate to $>1.8$
    \item Check if the second haar-like feature evaluate to $>.8$
    \item Check if the fourth haar-like feature evaluate to $>.35$
\end{itemize}

If all the above conditions are satisfied then the face is detected, else the blcok is rejected.

\textbf{Results}

Applied my face detection algorithm on image with faces and the results were as follows.

\begin{figure}[H]
    \centering
    \includegraphics[width=.9\textwidth]{imgs/face_detected.png}
    \caption{Face detection performance}
    \label{fig:2_mediapipe}
\end{figure}


Though all the four faces are detected, the face detection algorithm is not very accurate. There are too many false positives. This is because I have consider only three features which is not enough to describe the face.

\newpage

\newpage
\question

Imagine you’re monitoring pedestrian movement at a crosswalk. Your task is to track the direction and speed of pedestrians in a video using optical flow analysis. Use OpenCV’s built-in video vtest.avi, which simulates real-world pedestrian movement. \textbf{\textit{(Marks: 6 )}}

\begin{itemize}
    \item Using the Lucas-Kanade method, track specific points in the video to capture the movement of pedestrians.
    \item Visualize the direction of movement using arrows to indicate the flow direction at each point.
    \item Provide a brief summary: What patterns do you observe in pedestrian movement? Are there any areas where pedestrians tend to cluster or move faster?
\end{itemize}
\vspace{0.2in}
	\pointsdroppedatright 
\textbf{Answer:}

\textbf{Loading dataset}

I loaded 500 images of real faces and 500 images of spoof faces from the dataset.

\textbf{Raw pixel values}

The raw pixel values of the face images are used as features. I trained a Support Vector Machine (SVM) classifier on these raw pixel features to classify fake and real faces. The performance of the model on the dataset is portrayed in the code.

\textbf{Local Binary Patterns (LBP) features}

Local Binary Patterns (LBP) features are extracted from the face images for feature extraction. I trained an SVM classifier using the LBP features to classify fake and real faces. The performance of the model on the dataset is portrayed in the code.

\begin{figure}[H]
    \centering
    \includegraphics[width=.9\textwidth]{imgs/lbp.png}
    \caption{Local binary pattern}
    \label{fig:2_mediapipe}
\end{figure}

\newpage

\textbf{Edge methods}

Edge images are computed using the Prewitt and Sobel edge detectors. These edge images are used as input features independently to train an SVM classifier and classify fake and real faces. The performance of the model on the dataset is portrayed in the code.

\begin{figure}[H]
    \centering
    \includegraphics[width=.9\textwidth]{imgs/prewitt.png}
    \caption{Prewitt filtered image}
    \label{fig:2_mediapipe}
\end{figure}


\begin{figure}[H]
    \centering
    \includegraphics[width=.9\textwidth]{imgs/sobel.png}
    \caption{Sobel filtered image}
    \label{fig:2_mediapipe}
\end{figure}

\end{questions}

\end{document}